\begin{abstract}
\noindent Currently, testing a model of a sociotechnical system often does not include the uncertainty of action that comes with unreliable human actors. Human actors can be unreliable because their behaviour can be non-deterministic as a result of stresses that act upon them. This approach can be time consuming and produces models that are made particularly complicated by having this non-deterministic nature built into the model's mechanics.\par
A better approach might be to automate the human uncertainty involved by taking a deterministic "blueprint model", composed from the expected behaviour of the human actors, and made from some procedures that represent the actors' workflow. Then, when the human actors perform incorrectly, some code mutation technique is employed to enforce some random changes to their workflow, representing sociotechnical stresses. In this way, the concerns of the model's workflow and the actors' unreliability may be separated, and the model can be simple and human-readable while retaining the accuracy of more complex non-deterministic modelling techniques. 
\end{abstract}


%\noindent Currently, testing a model of sociotechnical systems includes modelling the potential difficulties inherent in that system and observing the effects if the system isn't followed correctly. There are some advantages to this approach, as a human observer might be able to pick out mistakes a human actor might introduce into the system. However, this approach can be time consuming and inconsistencies can be introduced when a human models the same system multiple times. \par
%A better approach might be to automate the human error involved in the execution of the sociotechnical model. This dissertation suggests a method for modelling sociotechnical systems directly in executable code, and using code fuzzing to simulate human inconsistency as a component of the model. We then provide some example models with this human error built in, and show some example applications of the technique to compare real-world workflows with the actors of the system being taken into account. We also show that the technique works as intended, that it fills a necessary gap in the sociotechnical modelling field, and that the technique has lots of room for future development, though its connections to \(\pi\)-calculus and as a useful method in its own right. \par