\begin{abstract}
\noindent Currently, testing a model of a sociotechnical system often does not include the uncertainty of action that comes with unreliable human actors. Human actors can be unreliable because their behaviour can be non-deterministic as a result of stresses that act upon them. This non-deterministic approach can be time consuming and produces models that are made particularly complicated by having this uncertain nature built into the model's mechanics.\par

\noindent A better approach might be to automate the human uncertainty involved by taking a deterministic "blueprint model", composed from the expected behaviour of the human actors, and made from some procedures that represent the actors' workflow. Then, when the human actors perform incorrectly, some code mutation technique is employed to enforce some random changes to their workflow, representing sociotechnical stresses. In this way, the concerns of the model's workflow and the actors' unreliability may be separated, and the model can be simple and human-readable while retaining the accuracy of more complex non-deterministic modelling techniques. \par

\noindent This report will explore the feasibility of mutating a sociotechnical workflow, so as to produce a "blueprint model" without this uncertainty built into the model, and will explore what such a model might look like. \par
\end{abstract}
