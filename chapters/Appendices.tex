\begin{appendices}
\crefalias{appendix}{appsec}
\appendix

% Add things as \chapters with a label beginning with "appendix:"
\chapter{Agile Flows}
\label{appendix:agile_flows}
Following are the flows used to test the Agile methodology: \par

\begin{pyglist}[language = python, caption={Flows used to test the Agile methodology}, listingname=\textbf{Code Listing} \comment{, fvset={frame=single,framerule=1pt}}, numbers=left]
@flow
def make_new_feature():
    create_feature()


def create_test_and_code():
    create_test_tdd()
    add_chunk_tdd()


@flow
def unit_test():
    run_tests()
    while not environment.resources["unit tests passing"]:
        fix_recent_feature()
        run_tests()


@flow
def fix_recent_feature():
    for feature in environment.resources['features']:
        for chunk in feature:
            fix_chunk(chunk)


@flow
def integration_test():
    perform_integration_tests()
    while not environment.resources["integration tests passing"]:
        unit_test()
        perform_integration_tests()


@flow
def user_acceptance_test():
    perform_user_acceptance_testing()
    while not environment.resources["user acceptance tests passing"]:
        unit_test()
        integration_test()
        perform_user_acceptance_testing()


@flow
@mutate(stressed)  # This gets changed to cannot_meet_deadlines for additional tests
def implement_feature():
    create_test_and_code()
    unit_test()
    integration_test()
    user_acceptance_test()


@flow
def implement_50_features():
    for i in range(50):
        make_new_feature()
        implement_feature()
\end{pyglist}

\chapter{Waterfall Flows}
\label{appendix:waterfall_flows}
Following are the flows used to test the Waterfall methodology: \par

\begin{pyglist}[language = python, caption={Flows used to test the Waterfall methodology}, listingname=\textbf{Code Listing} \comment{, fvset={frame=single,framerule=1pt}}, numbers=left]
@flow
def write_code():
    add_chunk_waterfall()


@flow
def unit_test():
    for chunk in environment.resources["features"]:
        if chunk.test is None: create_test_for_chunk(chunk)
    run_tests()
    while not environment.resources["unit tests passing"]:
        fix_codebase()
        run_tests()


@flow
def fix_codebase():
    for test in environment.resources["tests"]:
        if not test_passes(test):
            fix_chunk(test.chunk)


@flow
def integration_test():
    perform_integration_tests()
    while not environment.resources["integration tests passing"]:
        fix_codebase()
        perform_integration_tests()


@flow
def user_acceptance_test():
    perform_integration_tests()
    while not environment.resources["integration tests passing"]:
        fix_codebase()
        perform_integration_tests()
        integration_test()


@flow
@mutate(stressed) # This gets changed to cannot_meet_deadlines for additional tests
def create_product():
    for i in range(environment.resources["size of product in features"]):
        write_code()
    for i in range(environment.resources["size of product in features"]):
        unit_test()
    for i in range(environment.resources["size of product in features"]):
        integration_test()
    for i in range(environment.resources["size of product in features"]):
        user_acceptance_test()


@flow
def implement_50_features():
    environment.resources["size of product in features"] = 50
    create_product()
\end{pyglist}

\chapter{Mutator Functions}
\label{appendix:mutator_functions}
Following are the mutator functions used in this report's experiments.\par

\begin{pyglist}[language = python, caption={Mutator functions used in experiments}, listingname=\textbf{Code Listing} \comment{, fvset={frame=single,framerule=1pt}}, numbers=left]
def random_boolean():
    return random.choice([True, False])


def stressed(lines):
    if random_boolean():
        lines.remove(random.choice(lines))
    return lines


def cannot_meet_deadline(lines):
    if random_boolean():
        lines = lines[:random.randint(1, len(lines)-1)]
    return lines
\end{pyglist}

\chapter{Software Engineering atoms}
\label{appendix:atoms}
Following and the sociotechnical atoms used to power the previous flows.

\begin{pyglist}[language = python, caption={Atoms used in experiments}, listingname=\textbf{Code Listing} \comment{, fvset={frame=single,framerule=1pt}}, numbers=left]

@atom
def create_feature():
    environment.resources["features"].append([])
    environment.resources["current feature"] = len(environment.resources["features"])-1


@atom
def add_chunk(testing=False):
    environment.resources["time"] += 2
    feature = environment.resources["current feature"]
    chunk = dag.Chunk()

    # Add a test if it's necessary
    if testing:
        test = environment.resources["tests"][-1]
        chunk.test = test
        test.chunk = chunk
        environment.resources["tests"].append(test)

    # Record this chunk of code
    environment.resources["features"][feature].append(chunk)

    # Conditionally add a bug to the chunk
    if random.randint(3,5) is 4:
        bug = dag.Bug(chunk)
        environment.resources["bugs"].append(bug)

    # We're now working on the chunk we just created, so...
    environment.resources["current chunk"] = chunk


@atom
def add_chunk_waterfall(testing=False):
    environment.resources["time"] += 2
    chunk = dag.Chunk()

    # Add a test if it's necessary
    if testing:
        if len(environment.resources["tests"]) is not 0:
            test = environment.resources["tests"][-1]
            if test is not None:
                chunk.test = test
                test.chunk = chunk
                environment.resources["tests"].append(test)

    # Record this chunk of code
    environment.resources["features"].append(chunk)

    # Conditionally add a bug to the chunk
    if random.randint(3,5) is 4:
        bug = dag.Bug(chunk)
        environment.resources["bugs"].append(bug)

    # We're now working on the chunk we just created, so...
    environment.resources["current chunk"] = chunk


@atom
def add_chunk_tdd(testing=True):
    environment.resources["time"] += 2
    feature = environment.resources["current feature"]
    chunk = dag.Chunk()

    # Add a test if it's necessary
    if testing:
        if len(environment.resources["tests"]) is not 0:
            test = environment.resources["tests"][-1]
            if test is not None:
                chunk.test = test
                test.chunk = chunk
                environment.resources["tests"].append(test)

    # Record this chunk of code
    environment.resources["features"][feature].append(chunk)

    # Conditionally add a bug to the chunk
    if random.randint(3,5) is 4:
        bug = dag.Bug(chunk)
        environment.resources["bugs"].append(bug)

    # We're now working on the chunk we just created, so...
    environment.resources["current chunk"] = chunk


@atom
def create_test_for_chunk(chunk):
    environment.resources["time"] += 1
    if chunk.test is None:
        test = dag.Test(chunk)
        environment.resources["tests"].append(test)
        chunk.test = test

def create_test_tdd():
    environment.resources["time"] += 1
    test = dag.Test()
    environment.resources["tests"].append(test)


@atom
def fix_chunk(chunk=None):
    if chunk is None: chunk = environment.resources["current chunk"]

    # Iterate over bugs that affect the chunk and thrash until fixed
    for bug in environment.resources["bugs"]:
        while detects_bug(chunk.test, bug):
            environment.resources["time"] += cost_of_bug(bug)
            if random.randint(0, 5) is 4:
                remove_bug(bug)
                break


@atom
def perform_integration_tests():
    if len(environment.resources["bugs"]) == 0:
        environment.resources["integration tests passing"] = True
        return True
    number_of_messy_bugs = random.randint(0,len(environment.resources["bugs"]))
    environment.resources["integration tests passing"] = number_of_messy_bugs == 0

    return environment.resources["integration tests passing"]


@atom
def run_tests():
    environment.resources["unit tests passing"] = number_of_detected_bugs() == 0
    return environment.resources["unit tests passing"]


@atom
def perform_user_acceptance_testing():
    environment.resources["time"] += 1
    environment.resources["user acceptance tests passing"] = environment.resources["unit tests passing"]
    return environment.resources["user acceptance tests passing"]
\end{pyglist}

\chapter{Environment Layer}
\label{appendix:environment}
\begin{pyglist}[language = python, caption={Environment layer contents}, listingname=\textbf{Code Listing} \comment{, fvset={frame=single,framerule=1pt}}, numbers=left]
resources = {}
resources["time"] = 0
resources["seed"] = 0
resources["integration tests passing"] = False
resources["user acceptance tests passing"] = False
resources["unit tests passing"] = False
resources["tests"] = []
resources["successful deployment"] = False
resources["features"] = []  # To be a list of Chunk objects
resources["bugs"] = []  # To be a list of Bug objects
resources["mutating"] = True
resources["current chunk"] = None
resources["most recent test"] = None
resources["current bug"] = None
resources["current feature"] = 0
\end{pyglist}

\begin{pyglist}[language = python, caption={Classes used by sociotechnical environment}, listingname=\textbf{Code Listing} \comment{, fvset={frame=single,framerule=1pt}}, numbers=left]
import environment

class Vertex:
    id = 0

    def __init__(self):
        self.adj = []  # A list of vertices this vertex points to
        self.timeCreated = environment.resources["time"]
        self.id = Vertex.id
        Vertex.id += 1


    def add_adjacent(self, other_vertex):
        self.adj.append(other_vertex)

    def remove_adjacent(self, other_vertex):
        self.adj.remove(other_vertex)

    def adjacences(self):
        return self.adj

    def is_adjacent_to(self, other_vertex):
        return other_vertex in self.adj

    def age(self):
        return environment.resources["time"] - self.timeCreated



'''
    The chunks a bug affects are its adjacent vertices
    i.e. if a bug applies to a chunk, add the chunk as an adjacent vertex
'''
class Bug(Vertex):
    def __init__(self, buggy_chunk=None):
        Vertex.__init__(self)
        self.chunks = []
        if buggy_chunk is not None: self.chunks.append(buggy_chunk)

    def affects(self, chunk):
        for affectedChunk in self.chunks:
            if chunk.id == affectedChunk.id:
                return True
        return False


'''
    The bugs a chunk's tests will detect are the adjacent vertices
    i.e. if a chunk detects a bug, add the bug as an adjacent vertex
'''
class Chunk(Vertex):

    def __init__(self, test=None):
        Vertex.__init__(self)
        self.test = test

    def is_tested(self):
        return self.test is not None


'''
    A test that applies to a chunk or series of chunks
    If the test applies to more than one chunk, you should find them in the test's adjacency matrix
'''
class Test(Vertex):

    def __init__(self, chunk=None, works=True):
        Vertex.__init__(self)
        self.chunk = chunk
        self.works = works
\end{pyglist}

\chapter{Fuzzing Library}
\label{appendix:fuzzing_library}

\begin{pyglist}[language = python, caption={Complete fuzzing library}, listingname=\textbf{Code Listing} \comment{, fvset={frame=single,framerule=1pt}}, numbers=left]
import environment, random, ast, inspect

class ResourcesExpendedException(Exception):
    pass


class Mutator(ast.NodeTransformer):

    mutants_visited = 0

    # The mutation argument is a function that takes a list of lines and returns another list of lines.
    def __init__(self, mutation=lambda x: x, strip_decorators = True):
        self.strip_decorators = strip_decorators
        self.mutation = mutation

    #NOTE: This will work differently depending on whether the decorator takes arguments.
    def visit_FunctionDef(self, node):

        # Fix the randomisation to the environment
        random.seed(environment.resources["seed"])

        # Mutation algorithm!
        node.name += '_mod'   # so we don't overwrite Python's object caching
        # Remove decorators if we need to So we don't re-decorate when we run the mutated function, mutating recursively
        if self.strip_decorators:
            node.decorator_list = []
        # Mutate! self.mutation is a function that takes a list of line objects and returns a list of line objects.
        node.body = self.mutation(node.body)

        # Now that we've mutated, increment the necessary counters and parse the rest of the tree we're given
        Mutator.mutants_visited += 1
        environment.resources["seed"] += 1
        return self.generic_visit(node)


class mutate(object):

    cache = {}

    def __init__(self, mutation_type):
        self.mutation_type = mutation_type

    def __call__(self, func):
        def wrap(*args, **kwargs):

            if environment.resources["mutating"] is True:

                # Load function source from mutate.cache if available
                if func.func_name in mutate.cache.keys():
                    func_source = mutate.cache[func.func_name]
                else:
                    func_source = inspect.getsource(func)
                    mutate.cache[func.func_name] = func_source
                # Create function source
                func_source = ''.join(func_source) + '\n' + func.func_name + '_mod' + '()'
                # Mutate using the new mutator class
                mutator = Mutator(self.mutation_type)
                abstract_syntax_tree = ast.parse(func_source)
                mutated_func_uncompiled = mutator.visit(abstract_syntax_tree)
                mutated_func = func
                mutated_func.func_code = compile(mutated_func_uncompiled, inspect.getsourcefile(func), 'exec')
                mutate.cache[(func, self.mutation_type)] = mutated_func
                mutated_func(*args, **kwargs)
            else:
                func(*args, **kwargs)
        return wrap

    @staticmethod
    def reset():
        mutate.cache = {}
\end{pyglist}

\end{appendices}