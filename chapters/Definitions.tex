\begin{description}[align=right,labelwidth=3cm]
    \item [Sociotechnical System:] A sociotechnical system is a system that includes complex interactions between humans, machines, and the environment around them.\cite{Baxter2011}
    \item [Sociotechnical Variance:] Sociotechnical variance is the degree of change or non-determinism within a sociotechnical system.
    \item [Sociotechnical Environment:] The environment a sociotechnical system exists within, which can affect its emergent phenomena and its behaviour in general. A sociotechnical system typically affects its environment to at least some small degree.
    \item [Emergent Phenomena:] A perceived activity in a sociotechnical system that emerges out of the system's complexity, rather than being a part of the system explicitly added to the model. 
    \item [Sociotechnical Stress:] Some social or technical force influencing the activity of a sociotechnical system. This would often be either built into a model, or added after the fact, if a model is concerned with stresses. 
    \item [Code Fuzzing:] Altering source code or inputs to some black-box system to test what happens, often at random\cite{Miller1988}. For example, one might alter a sociotechnical system to make some changes at random to the model and perceive the outcome.
    \item [Mutation Testing:] Use of code fuzzing to verify tests. If a test passes code that has been mutated, the test might have a flaw, as it is passing code that should no longer perform its original purpose. 
    \item [Process Fuzzing:] Altering some process, as one might some code, to perceive the effect the change results in. This report focuses specifically on process fuzzing, but because of the similarities with Mutation Testing and Code Fuzzing, these terms are sometimes used here to mean Process Fuzzing. Therefore, if a fuzzing/mutation system is mentioned, and no meaning is explicitly stated, it should be assumed that Process Fuzzing is meant. 
    \item [Procedural Model:] A model made using procedures and programming code, rather than photographically or in a markup format.
    \item [Actor:] Anything which has agency within a sociotechnical system. Actors perform activities within the sociotechnical systems they operate in, and anything that has activity associated with it is an actor. 
    \item [Directed Acyclic Graph:] A Directed Graph is a graph\cite{Christofides:1975:GTA:1098653} with edges that point from some source node on the graph to some target node on the graph, and can only be traversed in the direction of the source to the target. A Directed Acyclic Graph is a Directed Graph with no Cycles, i.e. from any node it is impossible to traverse the graph such that that node is reached again.
    \item [DAG:] Abbreviation for Directed Acyclic Graph.
\end{description}