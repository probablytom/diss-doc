\chapter{Preliminaries}
\label{intro}
\pagenumbering{arabic}

\section{Introduction}
When a person wants to make a model of some real-world situation, lots of things need to be accounted for. Moreover, different models might be concerned with very different things. For example, a model of some physical system and the forces acting on it can be relatively easy to model, compared to a street on a busy day or the inner workings of a pancreas. There are also different types of complexity: where the pancreas plays host to several rather complex physiological interactions, other systems are less physical in nature. The busy street requires a degree of social modelling to model accurately, for example. \par

A model of a system with interactions between social and technical things, which also interact with their environments, is termed a \emph{sociotechnical} model. A sociotechnical model has unusual complexities involved in its modelling, such as people sharing information, or how they go about achieving a goal should a piece of equipment break. Unlike physical modelling problems, where the physics and mathematical laws governing how things act are well known, sociotechnical modelling has another degree of complexity in understanding the rules and guidelines that shape the behaviour of a person or group with their technology or environment. \par

\section{Context}
As a result of this high degree of complexity in sociotechnical modelling, the practice of creating sociotechnical models is notoriously difficult. This is especially true when trying to model the way that humans in these models act under stress. This is important to model for accuracy's sake, because these stresses can subtly affect the way people operate, and even small changes to sociotechnical models can greatly affect the product of a model\cite{Crabtree2000}. \par

As a result of the rapidly increasing degree of complexity, and the difficulties involved in simulating very important subtleties, sociotechnical modelling is a very difficult thing to do well. Regardless, sociotechnical modelling also has its uses, from mapping out the UK's National Health Service's IT systems\cite{Brennan2007} to ethnography\cite{Crabtree2000} and even risk analysis\cite{Storer2010}. \par

\section{Problem}
When sociotechnical modelling gets very complex, introducing the unreliability of its human actors becomes very difficult. it would be very useful to be able to model sociotechnical systems simply, and then introduce the unreliability of the human actors in the model as imperfections in the simulation. \par

There already exists a technique for introducing imperfections into programming code: \emph{code fuzzing}. Code fuzzing introduces changes to code to confirm that a programmer has written their tests appropriately. If the tests pass even when the code has been made bad, then the original code may contain bugs even if the tests pass. \par

This technique is interesting, because it suggests that it may be possible to produce these imperfections in a sociotechnical model, if the model is written in programming code that can be edited using existing methods. If those imperfections can be introduced in such a way that they are representative of the imperfections in human activity, then this might simplify the model created and help us to create more complex models without the overhead complexity of actor unreliability. \par

A method for creating this mutated model, representing human-like inconsistency in the code, and creating models suitable for mutation must therefore be made to determine whether code fuzzing can really simulate these inconsistencies, and whether this technique can be feasibly used. \par

\section{Terminology}
For clarification of terms within this report, some definitions of terms used have been provided: 
\label{introducing_technology}
\begin{description}[align=right,labelwidth=3cm]
    \item [Sociotechnical System:] A sociotechnical system is a system that includes complex interactions between humans, machines, and the environment around them.\cite{Baxter2011}
    \item [Sociotechnical Variance:] Sociotechnical variance is the degree of change or non-determinism within a sociotechnical system.
    \item [Sociotechnical Environment:] The environment a sociotechnical system exists within, which can affect its emergent phenomena and its behaviour in general. A sociotechnical system typically affects its environment to at least some small degree.
    \item [Emergent Phenomena:] A perceived activity in a sociotechnical system that emerges out of the system's complexity, rather than being a part of the system explicitly added to the model. 
    \item [Sociotechnical Stress:] Some social or technical force influencing the activity of a sociotechnical system. This would often be either built into a model, or added after the fact, if a model is concerned with stresses. 
    \item [Code Fuzzing:] Altering source code or inputs to some black-box system to test what happens, often at random\cite{Miller1988}. For example, one might alter a sociotechnical system to make some changes at random to the model and perceive the outcome.
    \item [Mutation Testing:] Use of code fuzzing to verify tests. If a test passes code that has been mutated, the test might have a flaw, as it is passing code that should no longer perform its original purpose. 
    \item [Process Fuzzing:] Altering some process, as one might some code, to perceive the effect the change results in. This report focuses specifically on process fuzzing, but because of the similarities with Mutation Testing and Code Fuzzing, these terms are sometimes used here to mean Process Fuzzing. Therefore, if a fuzzing/mutation system is mentioned, and no meaning is explicitly stated, it should be assumed that Process Fuzzing is meant. 
    \item [Procedural Model:] A model made using procedures and programming code, rather than photographically or in a markup format.
    \item [Actor:] Anything which has agency within a sociotechnical system. Actors perform activities within the sociotechnical systems they operate in, and anything that has activity associated with it is an actor. 
    \item [Directed Acyclic Graph:] A Directed Graph is a graph\cite{Christofides:1975:GTA:1098653} with edges that point from some source node on the graph to some target node on the graph, and can only be traversed in the direction of the source to the target. A Directed Acyclic Graph is a Directed Graph with no Cycles, i.e. from any node it is impossible to traverse the graph such that that node is reached again.
    \item [DAG:] Abbreviation for Directed Acyclic Graph.
\end{description}
