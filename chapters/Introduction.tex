\chapter{Preliminaries}
\label{intro}

%The project involves using a form of mutation testing called \emph{code fuzzing} to test models of sociotechnical systems under stress. Some definitions of terms used in this report: 
%\begin{itemize}
%\item \emph{Sociotechnical systems} are configurations of human actors and the technology they use. The human actors affect their environment and operate according to some workflow in an ideal case, though this workflow isn't always followed. This is due in part to the complexity of the system, and in part because the human actors do not exhibit reliable behaviour. Examples of sociotechnical systems are: \begin{itemize}
%\item Planes flown by pilots
%\item Teams of software developers
%\item A family organising a holiday online
%\item The same family booking their holiday through a travel website
%\end{itemize} 
%\item \emph{Stress} on a sociotechnical system is anything acting on the system which might make it change its behaviour; for example, a lack of money or conflict between actors. In particular, this project will be focusing on stresses which result in parts of the system's workflow being carried out without a part or many parts being completed. \par
%\item \emph{Mutation Testing} is the practice of modifying code and running it against tests to check how robustly the tests check the code's accuracy. Particularly, if the code is changed a little and the tests still pass, the tests may not catch incorrect behaviour in the code. \begin{itemize}
%\item \emph{Code Fuzzing} is a type of mutation testing which involves mixing lines of code up, removing lines, and truncating blocks of code, amongst other similar things. \par
%\end{itemize}
%\item The \emph{context} of a sociotechnical system can be considered its environment when it is practised. For example, the cities the actors reside within and the time the system is active at are both parts of the system's context.
%\end{itemize}



\section{Terminology}
For clarification of terms within this dissertation, some definitions of terms used have been provided: 
\label{introducing_technology}
\begin{description}[align=right,labelwidth=3cm]
    \item [Sociotechnical System:] A sociotechnical system is a system that includes complex interactions between humans, machines, and the environment around them.\cite{Baxter2011}
    \item [Sociotechnical Variance:] Sociotechnical variance is the degree of change or non-determinism within a sociotechnical system.
    \item [Sociotechnical Environment:] The environment a sociotechnical system exists within, which can affect its emergent phenomena and its behaviour in general. A sociotechnical system typically affects its environment to at least some small degree.
    \item [Emergent Phenomena:] A perceived activity in a sociotechnical system that emerges out of the system's complexity, rather than being a part of the system explicitly added to the model. 
    \item [Sociotechnical Stress:] Some social or technical force influencing the activity of a sociotechnical system. This would often be either built into a model, or added after the fact, if a model is concerned with stresses. 
    \item [Code Fuzzing:] Altering source code or inputs to some black-box system to test what happens, often at random\cite{Miller1988}. For example, one might alter a sociotechnical system to make some changes at random to the model and perceive the outcome.
    \item [Mutation Testing:] Use of code fuzzing to verify tests. If a test passes code that has been mutated, the test might have a flaw, as it is passing code that should no longer perform its original purpose. 
    \item [Process Fuzzing:] Altering some process, as one might some code, to perceive the effect the change results in. This report focuses specifically on process fuzzing, but because of the similarities with Mutation Testing and Code Fuzzing, these terms are sometimes used here to mean Process Fuzzing. Therefore, if a fuzzing/mutation system is mentioned, and no meaning is explicitly stated, it should be assumed that Process Fuzzing is meant. 
    \item [Procedural Model:] A model made using procedures and programming code, rather than photographically or in a markup format.
    \item [Actor:] Anything which has agency within a sociotechnical system. Actors perform activities within the sociotechnical systems they operate in, and anything that has activity associated with it is an actor. 
    \item [Directed Acyclic Graph:] A Directed Graph is a graph\cite{Christofides:1975:GTA:1098653} with edges that point from some source node on the graph to some target node on the graph, and can only be traversed in the direction of the source to the target. A Directed Acyclic Graph is a Directed Graph with no Cycles, i.e. from any node it is impossible to traverse the graph such that that node is reached again.
    \item [DAG:] Abbreviation for Directed Acyclic Graph.
\end{description}
