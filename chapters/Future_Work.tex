\chapter{Future Work} % For later!
\label{semantics}
\section{A mathematical sociotechnical model}
One piece of work which could be undertaken in the future would be to use the principles from this model to create a mathematically rigorous sociotechnical model. This alternative model would be parametrised by the outputs of the procedural model described through this dissertation. 
\subsection{Parametrisation}
Sociotechnical models have few to no concrete definitions in place. As a result, it can be difficult to discuss the properties of sociotechnical systems, as different people refer to different things. \\
However, the emergent properties of sociotechnical systems arise from two places:
\begin{itemize}
\item The generally deterministic running of the technical aspect of a system, which can break unexpectedly, leading to chaotic results
\item The seemingly random behaviour of the social component of the system, which leads to unpredictability by definition
\end{itemize}
However, using the atomic layer structure used to create the sociotechnical systems for these experiments, one could parametrise sociotechnical systems into properties of the social and technical aspects, which in turn have their own parameters to be defined.\\
Defining all of these relevant parameters would allow for characterisation of each sociotechnical atom by its affect on the different sociotechnical parameters, rather than they system's environment. As a result, each atom becomes a function which modifies values within the dimensions of the defined parameters. 

\subsection{Functions mapping to sociotechnical space}
If each atom modifies values within some sociotechnical dimensions, we can characterise an atom by specifying how it maps between points in sociotechnical space.\\
A natural extension of this is that flows can be defined by the composition of atoms' functions, such that all activities within the atomic layer model can be described as some mapping from one set of states within the sociotechnical space to another.\\
As a result, the ideal case described by a given sociotechnical layer model can be represented mathematically as a set of functions. However, the stress testing used to identify anomalies in sociotechnical systems then becomes alterations to this set of functions. One could, in fact, specify these alterations along these same sociotechnical dimensions, as these are the only values being changed. \\
Therefore, the output of testing the atomic model becomes a function space, where every point in the space is a behaviour characterised by how its emergent properties diverge from the properties of the ideal case. Moreover, this makes behaviour change a function mapping between points in the sociotechnical function space. The mapping of the behaviour change function is equivalent to the output of that change as modelled by code fuzzing, meaning that the fuzzing tools described in this dissertation is implicitly a tool for exploring this model.

\subsection{Creating a mathematically rigorous sociotechnical model}
\label{rigorous_sociotechnical_model}
Some future work to propose could be a realisation of this mathematical representation of a sociotechnical system. Defining mathematics and terms for the model would be an important step toward creating a single model for sociotechnical systems which can be analysed, for which tools are already available, and which unifies the jargon in the field such that researchers can discuss sociotechnical systems without the risk of ambiguity. 

\section{Middleware layers for submodel interaction}
\label{middleware-layers}
The interaction of models currently mostly relies on every detail of the model being worked into the system we're testing, because importing one set of activities into another doesn't account for the interplay between the two systems. \par
One must account for this interplay between systems, because missing even small details from a model leaves that model functionally incomplete: emergent phenomena from that small detail would be lost, and swathes of activity that should be modelled in a perfect case is lost in the incomplete one. \par
To combat this, one might create some middleware that allows two subsystems to interact by registering and referencing each other. In this way, activity from one system can influence another system, even though the two systems do not directly interact. \par
This middleware layer might also house the environment of the system, so that the environment is kept globally accessible to all procedures, and all subsystems that operate through the middleware interact with the same environment. This would keep the concerns of the modules bound together in one place, with all interactions with the modules of the larger system being together, whether those interactions took place directly between activities in the models or indirectly through changing their shared environment.

\section{Concurrent modelling}
\label{concurrent_modelling}
%Mention that different stresses occur when a sociotechnical system has multiple people working in it! 
%Mentioned in requirements of sociotechnical system \cref{planning_modelling_requirements}

\section{Fuzzing other models}
\label{fuzzing_other_models}


\section{More sophisticated fuzzing types}
\label{more_advanced_fuzzing}

